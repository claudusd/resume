%% start of file `template.tex'.
%% Copyright 2006-2013 Xavier Danaux (xdanaux@gmail.com).
%
% This work may be distributed and/or modified under the
% conditions of the LaTeX Project Public License version 1.3c,
% available at http://www.latex-project.org/lppl/.


\documentclass[11pt,a4paper]{moderncv}
% moderncv themes
\moderncvtheme[red]{claude}
% character encoding
\usepackage[utf8]{inputenc} % replace by the encoding you are using
% adjust the page margins
\usepackage[scale=0.9]{geometry}
%\usepackage{hypersetup}
%\hypersetup{colorlinks=false}
\recomputelengths % required when changes are made to page layout lengths
\usepackage{ifpdf}
\ifpdf
    \pdfinfo {
        /Author (Claude Dioudonnat)
        /Title (Claude Dioudonnat - Développeur Web et Admin Sys)
        /Subject (SUBJECT)
        /Keywords (KEYWORDS)
        /CreationDate (D:20191219135500)
    }
\fi

\usepackage[document]{ragged2e}

% personal data
\name{Claude}{Dioudonnat}
\title{Développeur Web \& Admin Sys}                               % optional, remove / comment the line if not wanted
\address{31 rue des neuf soleils}{63000 Clermont-Ferrand}{France}% optional, remove / comment the line if not wanted; the "postcode city" and "country" arguments can be omitted or provided empty
\phone[mobile]{+33~6~85~16~87~22}                   % optional, remove / comment the line if not wanted; the optional "type" of the phone can be "mobile" (default), "fixed" or "fax"
\email{claude@dioudonnat.fr}                               % optional, remove / comment the line if not wanted
%\homepage{www.johndoe.com}                         % optional, remove / comment the line if not wanted
%\social[linkedin]{john.doe}                        % optional, remove / comment the line if not wanted
\social[twitter]{ClaudusD}                             % optional, remove / comment the line if not wanted
\social[github]{claudusd}                              % optional, remove / comment the line if not wanted
\extrainfo{29 ans, Permis B}                 % optional, remove / comment the line if not wanted
%\photo[64pt][0.4pt]{picture}                       % optional, remove / comment the line if not wanted; '64pt' is the height the picture must be resized to, 0.4pt is the thickness of the frame around it (put it to 0pt for no frame) and 'picture' is the name of the picture file
%\quote{Some quote}                                 % optional, remove / comment the line if not wanted

%-------------------------------------------------------------------------------
% Content
%-------------------------------------------------------------------------------
\begin{document}
    \maketitle
    \section{Expériences professionnelles}
        \cventry{
          \textit{Depuis Avril 2019}
        }
        {Administrateur système}
        {LIMOS}
        {}
        {}
        {
          \justify
          Pour les besoins du laboratoire d'informatique Limos (UMR 6158) j'ai rejoins l'équipe pour \newline{}
          administrer les services de virtualisation (OpenStack et Proxmox) et mettre en place un \newline{}
          réseau virtuel (OpenVSwitch).
        }
        \cventry{
          \textit{Juin 2015 à Mars 2019\newline{}3 ans et 10 mois}
        }
        {Architecte Symfony2}
        {IT Network}
        {}
        {}
        {
          \justify
          Réalisation de projet avec Symfony2. Mise en place d'intégration et de déploiement continu avec\newline{}
          Travis-CI et Capistrano.
        }
        \cventry{\textit{Août 2014\newline{}10 mois}}{Développeur Symfony2}{Modis}{}{}{Développement d'un ERP immobilier sous Symfony2.}
        \cventry{\textit{Novembre 2013\newline{}9 mois}}{Ingenieur d'étude}{Université Blaise Pascal}{}{}{Maintenance du référentiel des formations (Java, XML, XSL-FO)}
        \cventry{\textit{Octobre 2012\newline{}1 an}}{Développeur Symfony2 et Android}{In My City}{}{}{Refonte du backoffice et de l'application Android}
        \cventry{\textit{Avril 2012\newline{} 5 mois}}{Stagiaire}{Centre Régional de Ressources Informatiques}{}{}{Développement d'un site web d'inscription aux activités sportives pour les étudiants \newline{}de Clermont-Ferrand. Travail sous Symfony2 et le SGBD Oracle.}

    \section{Etudes}
        \educationentry{\textit{2012 ~ 2013}}{Licence professionnelle en développement d'application mobile}{Iut Informatique, Université d'Auvergne}{Clermont-Ferrand}{}{}
        \educationentry{\textit{2009 ~ 2012}}{DUT Informatique}{Iut Informatique, Université d'Auvergne}{Clermont-Ferrand}{}{}
        \educationentry{\textit{2009}}{Baccalauréat scientifique (S) spécialité Sciences de l'ingénieur (SI)}{Lycée Charles et Adrien Dupuy}{Le Puy-en-Velay}{}{}

    \section{Enseignement}
        \cventry{
          \textit{Février, Mars, Juin et Juillet 2019}
        }
        {Formateur}
        {IPME et ADREC}
        {}
        {}
        {
          Formation sur l'intégration et le déploiement continu.
        }
        \cventry{
          \textit{Mars et Juillet 2019}
        }
        {Formateur}
        {Next Media et IPME}
        {}
        {}
        {
          Formation base de données (MYSQL et PostgreSQL)
        }
        \cventry{
          \textit{Octobre et Novembre 2018}
        }
        {Formateur}
        {Next Media}
        {}
        {}
        {
          Formation HTML5, PHP et Javascript.
        }
        \cventry{
          \textit{Juin 2018}
        }
        {Formateur}
        {IPME}
        {}
        {}
        {
            Formation sur les Design Pattern
        }
        \cventry{
          \textit{Janvier 2016 à 2020\newline{}6 semaines}
        }
        {Enseignant vacataire}
        {Université d'Auvergne}
        {}
        {}
        {
          Enseignement du javascript au étudiant de la licence web.\newline{}
          Préparation des cours et TP sur la norme ECMAScript 6, NodeJs, CSS
          responsive et ReactJS.
        }
        \cventry{
          \textit{Janvier 2015 et 2019\newline{}3 mois}
        }
        {Enseignant vacataire}
        {Université d'Auvergne}
        {}
        {}
        {
          Enseignement du javascript à 4 groupes d'étudiants de deuxième année.\newline{}
          Préparation des cours et TP.
        }
\pagebreak
    \section{Associatif}
        \cventry{\textit{Depuis Juillet 2016}}{Clermont'ech}{}{}{}
        {
          Membre du bureau, je participe à l'organisation des APIHours. Les
          APIHour sont \newline{}des soirées de conférence autour de la tech.
        }
        \cventry{\textit{Avril 2010 à Avril 2012\newline{}2 ans}}{ALPIC}{}{}{}
        {
          Président puis membre du bureau des étudiants de l'IUT Informatique.
        }

    \section{Compétences en informatique}
        \cvcomputer{\textit{Langages}} {PHP, Java, JavaScript, Python, Go}
            {\textit{SGBD \& NoSQL}} {MySQL, PostgreSQL, MongoDB, Redis}
        \cvcomputer{\textit{Framework \& SDK}} {Symfony2, React, Android }
            {\textit{Outils}} {Git, Vagrant, Capistrano}
        \cvcomputer{\textit{Côté Client}} {jQuery, HTML5, CSS, Sass, React}
            {\textit{Côté Serveur}} {Docker, Packer, Chef, Ansible, Terraform, Proxmox}
        \cvcomputer{\textit{Methodologie}} {TDD, DevOps, CI, CD}
            {\textit{Analyse}} {Merise, UML, Design Pattern}
\end{document}
